\documentclass[a4paper, 12pt]{extarticle}

% used for custom enumarations
\usepackage{enumerate}
\usepackage[shortlabels]{enumitem}

% lets tables span over page
\usepackage{longtable}
\usepackage{tabularx}

% German language
\usepackage[ngerman]{babel}
\usepackage[utf8]{inputenc}
\usepackage[T1]{fontenc}

\usepackage{fancyhdr}
\usepackage{enumitem}
\usepackage{amsmath}
\usepackage{amssymb}

\usepackage[margin=1in]{geometry}
\setlength\parindent{0pt}

\pagestyle{fancy}
\fancyhf{}
\rhead{Cheatsheet}
\lhead{Algebra}
\cfoot{\thepage}

\begin{document}

\section*{Tautologien}

\subsection*{Allgemeine Tautologien}

\begin{tabular}{p{8cm}|p{7cm}}
Dopplete Negation: & $\lnot \lnot A \Leftrightarrow A$\\
\hline
Idempotenz der Konjunktion & $A \land A \Leftrightarrow A$\\
\hline
Idempotenz der Disjunktion & $A \lor A \Leftrightarrow A$\\
\hline
Gesetz des Widerspruchs & $\lnot (A \land \lnot A) \Leftrightarrow 1$\\
\hline
Gesetz des ausgeschlossenen Dritten & $A \lor \lnot A \Leftrightarrow 1$\\
\hline
Konjunktion mit 0 & $A \land 0 \Leftrightarrow 0$\\
\hline
Konjunktion mit 1 & $A \land 1 \Leftrightarrow A$\\
\hline
Disjunktion mit 0 & $A \lor 0 \Leftrightarrow A$\\
\hline
Disjunktion mit 1 & $A \lor 1 \Leftrightarrow 1$
\end{tabular}

\subsection*{Rechenregeln}

\begin{tabular}{p{8cm}|p{7cm}}
Kommutativgesetz & $A \land B \Leftrightarrow B \land A$\\
Kommutativgesetz & $A \lor B \Leftrightarrow B \lor A$\\
\hline
Assoziativgesetz & $(A \land B) \land C \Leftrightarrow A \land (B \land C)$\\
Assoziativgesetz & $(A \lor B) \lor C \Leftrightarrow A \lor (B \lor C)$\\
\hline
Distributivgesetz & $A \land (B \lor C) \Leftrightarrow (A \land B) \lor (A \land C)$\\
Distributivgesetz & $A \lor (B \land C) \Leftrightarrow (A \lor B) \land (A \lor C)$\\
\hline
Absorptionsgesetz & $A \land (A \lor B) \Leftrightarrow A$\\
Absorptionsgesetz & $A \lor (A \land B) \Leftrightarrow A$\\
\hline
Sätze von De Morgan zur & $\lnot (A \land B) \Leftrightarrow \lnot A \lor \lnot B$\\
Vereinigung von Aussagen & $\lnot (A \lor B) \Leftrightarrow \lnot A \land \lnot B$\\
\hline
Kettenschlussregel & $(A \Rightarrow B) \land (B \Rightarrow C) \Rightarrow (A \Rightarrow C)$\\
Kontrapositionssatz & $(A \Rightarrow B) \Leftrightarrow (\lnot B \Rightarrow \lnot A)$\\
\hline
Satz v. modus (ponendo) ponens & $(A \land (A \Rightarrow B) \Rightarrow B)$\\
Satz v. modus (tollendo) tollens & $(A \Rightarrow B) \land \lnot B \Rightarrow \lnot A$\\
Satz v. modus tollendo ponens & $(A \lor B) \land \lnot A \Rightarrow B$\\
Satz v. modus ponendo tollens & $\lnot (A \land B) \land A \Rightarrow \lnot B$
\end{tabular}

\end{document}