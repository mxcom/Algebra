\section*{Aufgabe 4}

Es sei $M = {1,2,3,4,5}$. Die Funktionen $f,g : M \rightarrow M$ seien gegeben durch\\

$f = \begin{pmatrix}
1 & 2 & 3 & 4 & 5\\
2 & 1 & 4 & 5 & 3\\
\end{pmatrix} \text{ bzw. } g = \begin{pmatrix}
1 & 2 & 3 & 4 & 5\\
2 & 3 & 4 & 1 & 5\\
\end{pmatrix}$.\\~\\

a) Man bestimmte $f^{-1}$, $g^{-1}$, $f \circ g$, $g \circ f$.\\

$f^{-1} = \begin{pmatrix}
1 & 2 & 3 & 4 & 5\\
2 & 1 & 5 & 3 & 4\\
\end{pmatrix}, g^{-1} = \begin{pmatrix}
1 & 2 & 3 & 4 & 5\\
4 & 1 & 2 & 3 & 5\\
\end{pmatrix}$\\~\\

$f \circ g = \begin{pmatrix}
1 & 2 & 3 & 4 & 5\\
1 & 4 & 5 & 2 & 3\\
\end{pmatrix}, g \circ f = \begin{pmatrix}
1 & 2 & 3 & 4 & 5\\
3 & 2 & 1 & 5 & 4\\
\end{pmatrix}$\\~\\

b) Mit $f^{(k)} = f \circ f \circ \dots \circ f$ ($k$-mal) bezeichnen wir die $k$-fache Komposition von $f$ mit sich selbst. Man bestimmte die kleinste Zahlen $k,l \in \mathbb{N}$, sodass gilt: $f^{(k)} = id_M$ und $g^{(l)} = id_M$.\\

$\hspace{0.4cm} f = \begin{pmatrix}
1 & 2 & 3 & 4 & 5\\
2 & 1 & 4 & 5 & 3\\
\end{pmatrix}, \hspace{2.4cm} g = \begin{pmatrix}
1 & 2 & 3 & 4 & 5\\
2 & 3 & 4 & 1 & 5\\
\end{pmatrix}$\\~\\

$f^{(2)} = \begin{pmatrix}
1 & 2 & 3 & 4 & 5\\
1 & 2 & 5 & 3 & 4\\
\end{pmatrix}, \hspace{2cm} g^{(2)} = \begin{pmatrix}
1 & 2 & 3 & 4 & 5\\
3 & 4 & 1 & 2 & 5\\
\end{pmatrix}$\\~\\

$f^{(3)} = \begin{pmatrix}
1 & 2 & 3 & 4 & 5\\
2 & 1 & 3 & 4 & 5\\
\end{pmatrix}, \hspace{2cm} g^{(3)} = \begin{pmatrix}
1 & 2 & 3 & 4 & 5\\
4 & 1 & 2 & 3 & 5\\
\end{pmatrix}$\\~\\

$f^{(4)} = \begin{pmatrix}
1 & 2 & 3 & 4 & 5\\
1 & 2 & 4 & 5 & 3\\
\end{pmatrix}, \hspace{2cm} g^{(4)} = \begin{pmatrix}
1 & 2 & 3 & 4 & 5\\
1 & 2 & 3 & 4 & 5\\
\end{pmatrix}$\\~\\

$f^{(5)} = \begin{pmatrix}
1 & 2 & 3 & 4 & 5\\
2 & 1 & 5 & 3 & 4\\
\end{pmatrix}$\\~\\

$f^{(6)} = \begin{pmatrix}
1 & 2 & 3 & 4 & 5\\
1 & 2 & 3 & 4 & 5\\
\end{pmatrix}$\\~\\