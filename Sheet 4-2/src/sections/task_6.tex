\section*{Aufgabe 6}

Gegeben seien Mengen $A$, $B$ und $C$ sowie Funktionen $f: A \rightarrow B$ und $g: B \rightarrow C$.\\

a) Zeigen Sie:

\begin{enumerate}
\item[(i)] $g \circ f$ surjektiv $\Rightarrow$ $g$ surjektiv.
\end{enumerate}

\textit{Annahme:}\\
- \textit{$g \circ f: A \rightarrow C$ ist surjektiv}\\
- \textit{$f: A \rightarrow B$ ist surjektiv}\\

\textit{Zu Zeigen:}\\
\textit{Damit $g$ surjektiv ist muss für jedes $c \in C$ ein Element $b \in B$ existieren, sodass $g(b) = c$.}\\

\textit{Da $g \circ f$ surjektiv ist, gibt es für jedes $c \in C$ ein $a \in A$, sodass $(g \circ f)(a) = c$. Da $f$ surjektiv ist, gibt es für jedes $b \in B$ ein a $a \in A$, sodass $f(a) = b$. Daraus folgt, dass für jedes das für jedes $c \in C$ ein Element $b \in B$ existieren muss, sodass $g(b) = c$ ist. Damit ist $g$ surjektiv.}\\

\begin{enumerate}
\item[(ii)] $g \circ f$ injektiv $\Rightarrow$ $f$ injektiv.
\end{enumerate}

\textit{Zu Zeigen:}\\
\textit{Damit $f$ injektiv ist muss darf jedem $b \in B$ höchstens ein $a \in A$ zugeordnet werden.}\\
