\section*{Aufgabe 3}

Es sei $f: \mathbb{R} \rightarrow \mathbb{R}^2, \ f(x) = (\frac{x}{1+x^2}, x^2)$ und $g: \mathbb{R}^2 \rightarrow \mathbb{R}, \ g(x,y) = x(y+1)$.\\

a) Berechnen Sie $f \circ g$ und $g \circ f$.\\

$f \circ g : \mathbb{R}^2 \rightarrow \mathbb{R}^2, \ f(g(x,y)) = ((\frac{x(y+1)}{1+(x(y+1))^2}), (x(y+1))^2)$\\~\\

$g \circ f : \mathbb{R} \rightarrow \mathbb{R}, \ g(\frac{x}{1+x^2},x^2) = \frac{x}{1+x^2}(x^2+1)$\\~\\

b) Berechnen Sie $f(-1)$, $f(2)$, $g(1,2)$, $g(-1,1)$.

\hspace{0.54cm}Berechnen Sie ferner $(f \circ g)(1,1)$ und $(f \circ g)(2,0)$, sowie $(g \circ f)(3)$ und $(g \circ f)(0)$\\

$f(-1) = (\frac{-1}{1+(-1)^2}, -1^2) = (- \frac{1}{2}, 1)$\\~\\

$f(2) = (\frac{2}{1+2^2}, 2^2) = (\frac{2}{5}, 4)$\\~\\

$g(1,2) = 1(2+1) = 3$\\~\\

$g(-1,1) = -1(1+1) = -2$\\~\\

$(f \circ g)(1,1) = ((\frac{1(1+1)}{1+(1(1+1))^2}), (1(1+1))^2) = (\frac{2}{5}, 4)$\\~\\

$(f \circ g)(2,0) = ((\frac{2(0+1)}{1+(2(0+1))^2}), (2(0+1))^2) = (\frac{2}{5}, 4)$\\~\\

$(g \circ f)(3) = \frac{3}{1+3^2}(3^2+1) = \frac{3}{10}(10) = \frac{10}{10} = 1$\\~\\

$(g \circ f)(0) = \frac{0}{1+0^2}(0^2+1) = 0$\\~\\

c) Welche der Funktionen $f$, $g$, $f \circ g$ und $g \circ f$ sind injektiv bzw. surjektiv bzw. bijektiv?

\newpage
