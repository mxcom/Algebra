\section*{Aufgabe 3}

Gegeben sei ein kommutativer Ring mit Einselement $(A, \oplus, \otimes)$.\\
Auf $B = A \times A$ definieren wir zwei Verknüpfungen, die wir wieder mit $\oplus$ und $\otimes$ bezeichnen:\\
$(a_1, a_2) \oplus (b_1, b_2) := (a_1 + b_1, a_2 + b_2)$\\
$(a_1, a_2) \otimes (b_1, b_2) := (a_1 \ast b_1, a_2 \ast b_2)$

\begin{itemize}[leftmargin=*, label={a)}]
\item Zeigen Sie, dass $B$ mit diesen Verknüpfungen wieder ein kommutativer Ring mit Einselement
ist.
\end{itemize}

\textit{Um zu zeigen, dass $B$ mit den gegebenen Verknüpfungen ein kommutativer Ring mit Einselement ist, muss man zeigen, dass $B$ die folgenden Eigenschaften erfüllt:}

\begin{enumerate}[leftmargin=*]
\item \textit{Assoziativität der Addition: $(a \oplus b) \oplus c = a \oplus (b \oplus c)$ für alle $a$, $b$, $c$ in $B$.}
\item \textit{Kommutativität der Addition: $a \oplus b = b \oplus a$ für alle $a$, $b$ in $B$.}
\item \textit{Existenz eines neutralen Elements bei der Addition: Es gibt ein Element $0$ in $B$, so dass $a \oplus 0 = a$ für alle $a$ in $B$.}
\item \textit{Existenz eines inversen Elements bei der Addition: Für jedes $a$ in $B$ gibt es ein Element $a^{-1}$ in $B$, so dass $a \oplus (a^{-1}) = 0$.}
\item \textit{Assoziativität der Multiplikation: $(a \otimes b) \otimes c = a \otimes (b \otimes c)$ für alle $a$, $b$, $c$ in $B$.}
\item \textit{Kommutativität der Multiplikation: $a \otimes b = b \otimes a$ für alle $a$, $b$ in $B$.}
\item \textit{Existenz eines neutralen Elements bei der Multiplikation: Es gibt ein Element $1$ in $B$, so dass $a \otimes 1 = a$ für alle $a$ in $B$.}
\item \textit{Distributivität: $a \otimes (b \oplus c) = (a \otimes b) \oplus (a \otimes c)$ und $(b \oplus c) \otimes a = (b \otimes a) \oplus (c \otimes a)$ für alle $a$, $b$, $c$ in $B$.}
\end{enumerate}

\textit{Anhand der angegebenen Verknüpfungen sieht man, dass alle oben genannten Eigenschaften erfüllt sind.}

\begin{itemize}[leftmargin=*]
\item \textit{Die Assoziativität der Addition und Multiplikation erfüllt, da es sich um die gleiche Assoziativität wie in $A$ handelt.}
\item \textit{Die Kommutativität der Addition und Multiplikation erfüllt, da es sich um die gleiche Kommutativität wie in $A$ handelt.}
\item \textit{Der neutrale Element bei der Addition ist $(0,0)$ und bei der Multiplikation ist $(1,1)$}
\item \textit{Für jedes $(a_1,a_2)$ gibt es ein inverses Element $(a_1^{-1}, a_2^{-1})$}
\item \textit{Die Distributivität erfüllt, da es sich um die gleiche Distributivität wie in $A$ handelt.}
\end{itemize}

\newpage

\begin{itemize}[leftmargin=*, label={b)}]
\item Bestimmen Sie alle Einheiten von $B$.
\end{itemize}

\textit{Gesucht sind alle Elemente für die gilt: $(a_1,a_2) \otimes (b_1,b_2) = (1,1)$.}\\

$Einheiten \ von \ B: (1,1)$\\

\begin{itemize}[leftmargin=*, label={c)}]
\item Bestimmen Sie alle Nullteiler von $B$.
\end{itemize}

\textit{Gesucht sind Elemente für die gilt $x,y \neq (0,0)$ mit $x \otimes y = (0,0)$.}\\

$Nullteiler \ von \ B: (1,0), (0,1)$\\

\begin{itemize}[leftmargin=*, label={d)}]
\item Zeigen Sie, dass das Assoziativgesetz nicht gilt, wenn Sie statt dessen die Definition $(a_1, a_2) \oplus (b_1, b_2) := (a_1 + a_2, b_1 + b_2)$ verwendet hätten
\end{itemize}

\textit{Damit die Assioziativät der Addition gilt muss $\forall a,b,c \in B: (a \oplus b) \oplus c = a \oplus (b \oplus c)$ gelten. Man muss also drei $a,b,c \in B$ finden für die die Definition nicht gilt.}\\

\textit{Sei $(1,2), (3,4), (5,6) \in B$ dann muss $((1,2) \oplus (3,4)) \oplus (5,6) = (1,2) \oplus ((3,4) \oplus (5,6))$.}\\

(i)
\begin{align*}
((1,2) \oplus (3,4)) \oplus (5,6)&=\\
((1 + 2, 3 + 4)) \oplus (5,6)&=\\
(3, 7) \oplus (5,6)&=\\
(3 + 7, 5 + 6)&=\\
(10,11)
\end{align*}

(ii)
\begin{align*}
(1,2) \oplus ((3,4) \oplus (5,6))&=\\
(1,2) \oplus ((3 + 4, 5 + 6))&=\\
(1,2) \oplus (7, 11)&=\\
(1+2, 7+11)&=\\
(3, 18)
\end{align*}

\textit{$(10,11) \neq (3, 18)$ somit gilt die Definition der Assioziativität nicht für alle Elemente $a,b,c \in B$.}

\newpage