\section*{Aufgabe 4}

Wir betrachten den Ring der Polynome $\mathbb{Q}[x]$ über dem Körper der rationalen Zahlen. Dieser Ring ist ein ``euklidischer Ring'' bezüglich der Grad-Funktion, d.h.:

\begin{itemize}[label=""]
\item Für je zwei Polynome $f, g \in \mathbb{Q}[x]$ mit $g \neq 0$ gibt es Polynome $p, r \in \mathbb{Q}[x]$ mit\\
$f = p * g + r$, wobei gilt: $grad(r) < grad(g)$\\
(Polynomdivision mit Rest)
\end{itemize}

Für die folgenden Polynome $f$, $g$ berechne man jeweils die zugehörigen $p$, $q$:

\begin{itemize}[label={a)}, leftmargin=*]
\item $f = 3x^4 - 2x^2 + x + 1, g = x + 2$
\end{itemize}

\polylongdiv[style=C]{3x^4-2x^2+x+1}{x+2}

\begin{itemize}[leftmargin=*]
\item $p = 3x^3 - 6x^2 + 10x - 19$
\item $r = 39$
\end{itemize}\


\begin{itemize}[label={b)}, leftmargin=*]
\item $f = x^5 - x^4 + x^3 - x^2 + x - 1, g = x^3 + x^2 + x + 1$
\end{itemize}

\polylongdiv[style=C]{x^5-x^4+x^3-x^2+x-1}{x^3+x^2+1}

\begin{itemize}[leftmargin=*]
\item $p = x^2 -2x + 3$
\item $r = -5x^2 +3x -4$
\end{itemize}\


