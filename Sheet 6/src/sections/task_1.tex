\section*{Aufgabe 1}

Wir betrachten die Ringe $(\mathbb{Z}_n, \oplus, \otimes)$

\begin{itemize}[leftmargin=*, label={a)}]
\item Für $n = 15$ bestimmen Sie die Einheitengruppe $(\mathbb{Z}^{*}_{15}, \otimes)$\\
Geben Sie die Elemente dieser Gruppe an und bestimmen Sie die Gruppentafel.\\
Ist diese Gruppe zyklisch?\\
Geben Sie die Ordnungen aller Elemente an.
\end{itemize}

$\mathbb{Z}^{*}_{15} = \{1, 2, 4, 7, 8, 11, 13, 14\} \rightarrow$ \textit{Alle Elemente wo Rest gleich dem Neutralem Element ist (also 1), da Elemente invertierbar seien müssen}\\

\begin{table}[h]
\centering
\begin{tabular}{c|cccccccc}
$\otimes$ & 1 & 2 & 4 & 7 & 8 & 11 & 13 & 14\\
\hline
1  &  1 &  2 &  4 &  7 &  8 & 11 & 13 & 14\\
2  &  2 &  4 &  8 & 14 &  1 &  7 & 11 & 13\\
4  &  4 &  8 &  1 & 13 &  2 & 14 &  7 & 11\\
7  &  7 & 14 & 13 &  4 & 11 &  2 &  1 & 8\\
8  &  8 &  1 &  2 & 11 &  4 & 13 & 14 & 7\\
11 & 11 &  7 & 14 &  2 & 13 &  1 &  8 & 4\\
13 & 13 & 11 &  7 &  1 & 14 &  8 &  4 & 2\\
14 & 14 & 13 & 11 &  8 &  7 &  4 &  2 & 1
\end{tabular}
\end{table}

\textit{Zyklisch bedeutet, dass die Potenzen eines Elementes aus einer Menge, alle Elemente der Menge ergibt. Man muss also ein Erzeuger finden. Die Einheitengruppe $\mathbb{Z}^{*}_{15}$ ist also nicht zyklisch, da kein Erzeuger existiert.}\\

\textit{Die Ordnung eines Elements ist das Element, dass bei der Verknüpfung das neutrale Element ergibt.}

\begin{itemize}
\item $\mathcal{O}(1) = 1$
\item $\mathcal{O}(2) = 8$
\item $\mathcal{O}(4) = 4$
\item $\mathcal{O}(7) = 13$
\item $\mathcal{O}(8) = 2$
\item $\mathcal{O}(11) = 11$
\item $\mathcal{O}(13) = 7$
\item $\mathcal{O}(14) = 14$
\end{itemize}

\newpage

\begin{itemize}[leftmargin=*, label={b)}]
\item Haben die folgenden Gleichungen in $\mathbb{Z}_{15}$ eine Lösung?\\
Falls eine Lösung existiert, bestimmen Sie diese!\\
$4 \otimes x \oplus 6 = 8$\\
$10 \otimes x \oplus 3 = 4$
\end{itemize}

(i)
\begin{align*}
4 \otimes x \oplus 6 &= 8\\
(4 \otimes x) \oplus 6 &= 8 &&| \oplus (\ominus 6)\\
(4 \otimes x) \oplus 6 \oplus (\ominus 6) &= 8 \oplus (\ominus 6)\\
(4 \otimes x) \oplus \cancel{6 \oplus (\ominus 6)} &= 8 \oplus \underbrace{(\ominus 6)}_{9}\\
(4 \otimes x) \oplus 0 &= 8 \oplus 9\\
4 \otimes x &= 2 &&| 4^{-1} \otimes\\
4^{-1} \otimes 4 \otimes x &= 4^{-1} \otimes 2\\
\cancel{4^{-1} \otimes 4} \otimes x &= \underbrace{4^{-1}}_{4} \otimes 2\\
x &= 4 \otimes 2\\
x &= 8
\end{align*}

(ii)\\

\textit{$10 \otimes x \oplus 3 = 4$ hat keine Lösung. $10 \otimes x$ müsste $1$ sein damit $1 \oplus 3 = 4$ ergeben kann. Es existiert allerdings kein $x$ damit dies der Fall ist.}\\

\begin{itemize}[leftmargin=*, label={c)}]
\item Warum ist $\mathbb{Z}_{11}$ ein Körper?\\
Bestimmen Sie für jedes Element in $\mathbb{Z}_{11} \backslash \{0\}$ das multiplikative inverse Element.
\end{itemize}

$\mathbb{Z}_{11} = \{0, 1, 2, 3, 4, 5, 6, 7, 8, 9, 10, 11\}$.
\begin{enumerate}[leftmargin=*]
	\item $(\mathbb{Z}_{11}, +)$ \textit{ist eine abelsche Gruppe.}
	\begin{itemize}
		\item \textit{Assoziativ} $a + b = b + a$
		\item \textit{neutrales Element (0) existiert}
		\item \textit{Für alle Elemente existiert inverses Element}
		\item \textit{kommutativ} $(a + b) + c = a + (b + c)$
	\end{itemize}
	\item $(\mathbb{Z}_{11} \backslash \{0\}, \cdot)$
		\begin{itemize}
		\item \textit{Assoziativ} $a \cdot b = b \cdot a$
		\item \textit{neutrales Element (1) existiert}
		\item \textit{Für alle Elemente existiert inverses Element}
		\item \textit{kommutativ} $(a \cdot b) \cdot c = a \cdot (b \cdot c)$
	\end{itemize}
	\item \textit{Distributivgesetze gelten}
\end{enumerate}

\newpage

\textit{Multiplikative Inverse:}

\begin{itemize}
	\item $1^{-1} = 1$
	\item $2^{-1} = 6$
	\item $3^{-1} = 4$
	\item $4^{-1} = 3$
	\item $5^{-1} = 9$
	\item $6^{-1} = 2$
	\item $7^{-1} = 8$
	\item $8^{-1} = 7$
	\item $9^{-1} = 5$
	\item $10^{-1} = 10$
\end{itemize}

\newpage

\begin{itemize}[leftmargin=*, label={d)}]
\item Lösen Sie in $\mathbb{Z}_{11}$ die folgenden Gleichungen:\\
$4 \otimes x \oplus 6 = 8$\\
$10 \otimes x \oplus 3 = 4$
\end{itemize}

(i)
\begin{align*}
4 \otimes x \oplus 6 &= 8\\
(4 \otimes x) \oplus 6 &= 8 &&| \oplus (\ominus 6)\\
(4 \otimes x) \oplus 6 \oplus (\ominus 6) &= 8 \oplus (\ominus 6)\\
(4 \otimes x) \oplus \cancel{6 \oplus (\ominus 6)} &= 8 \oplus \underbrace{(\ominus 6)}_{5}\\
(4 \otimes x) \oplus 0 &= 8 \oplus 5\\
4 \otimes x &= 2 &&| 4^{-1} \otimes \\
4^{-1} \otimes 4 \otimes x &= 4^{-1} \otimes 2\\
\cancel{4^{-1} \otimes 4} \otimes x &= \underbrace{4^{-1}}_{3} \otimes 2\\
x &= 3 \otimes 2\\
x &= 6
\end{align*}

(ii)
\begin{align*}
10 \otimes x \oplus 3 &= 4\\
(10 \otimes x) \oplus 3 &= 4 &&| \oplus (\ominus 3)\\
(10 \otimes x) \oplus 3 \oplus (\ominus 3) &= 4 \oplus (\ominus 3)\\
(10 \otimes x) \oplus \cancel{3 \oplus (\ominus 3)} &= 4 \oplus \underbrace{(\ominus 3)}_{8}\\
(10 \otimes x) \oplus 0 &= 4 \oplus 8\\
10 \otimes x &= 1 &&| 10^{-1} \otimes\\
10^{-1} \otimes 10 \otimes x &= 10^{-1} \otimes 1\\
\cancel{10^{-1} \otimes 10} \otimes x &= \underbrace{10^{-1}}_{10} \otimes 1\\
x &= 10 \otimes 1\\
x &= 10
\end{align*}
\newpage

