\section*{Aufgabe 1}

Entscheiden Sie, welche der folgenden (sprachlichen) Sätze man als Aussagen bezeichnen könnte:\\

a) Die Sonne ist ein Fixstern und der Mond ein Planet.\\

\textit{Aussage}\\

b) Auf dem Planeten Alpha Centauri gibt es Eisen.\\

\textit{Aussage (Begründung: Unabhängig vom Wissensstand ob es wirklich so ist, können wir 'wahr' oder 'falsch' zuordnen)}\\

c) $3 \cdot 3 = 10$ .\\

\textit{Keine Aussage (Begründung: Es ist kein sprachlicher Satz)}\\

d) Herr Ober, bitte zahlen, und schreiben Sie auch eine Rechnung aus!\\

\textit{Keine Aussage}\\

e) Je zwei Geraden schneiden sich in einem Punkt.\\

\textit{Aussage}\\

f) Es werde Licht.\\

\textit{Keine Aussage}\\

g) Es gilt $x^2 - 2x + 5x = 0$.\\

\textit{Keine Aussage (Begründung: Wir wissen nicht was $x$ ist, somit ist es eine Aussageform und keine Aussage)}\\

h) Alle (FH-)Lehrer sind faule Säcke, bekommen aber ein hohes Gehalt.\\

\textit{Aussage (Begründung: 'wahr' oder 'falsch' sind logisch zuordbar)}\\

i) Wie lange noch wirst Du unsere Geduld missbrauchen?\\

\textit{Keine Aussage}\\

j) Die Gleichung $x^n + y^n = z^n$ hat für $n \geq 3$ in den positiven ganzen Zahlen keine Lösung.\\

\textit{Aussage}\\

k) Für jede reelle Zahl $a$ ist die Gleichung $ax + b = 0$ lösbar.\\

\textit{Keine Aussage (Begründung: $b$ ist nicht gegeben)}\\

l) Für je zwei Zahlen $a$ und $b$ ist die Gleichung $ax + b = 0$ lösbar.\\

\textit{Aussage (Begründung: $a$ und $b$ sind gegeben)}\\
