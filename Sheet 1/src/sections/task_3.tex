\section*{Aufgabe 3}

Versuchen Sie, für jede der Aussagen aus Aufgabe 2 die Negation zu bilden!\\

Dabei ist es wichtig, eine Formulierung zu finden, die logisch genau den Gehalt der Aussage 'Es gilt nicht, dass \dots ' hat, die aber so formuliert ist, wie ein 'normaler' Mensch (also kein Mathematiker!?!) sie aussprechen würde.\\

a) Köln und Düsseldorf liegen am Rhein.\\

\textit{Weder Köln noch Düsseldorf liegen am Rhein.}\\

b) Eine der Städte Frankfurt, Köln, Dresden und Berlin liegt an der Elbe.\\

\textit{Bis auf eine der Städte Frankfurt, Köln, Dresden und Berlin liegen alle an der Elbe.}\\

c) Wenn es regnet, werden die Straßen nass und der Himmel verdunkelt sich.\\

\textit{Regnet es nicht, werden die Straße nicht nass und der Himmel verdunkelt sich nicht.}\\

d) Ist $p$ eine Primzahl und teilt $p$ das Produkt $xy$ der zwei Zahlen $x$ und $y$, so teilt $p$ auch einen der beiden Faktoren $x$ oder $y$.\\

\textit{Ist $p$ keine Primzahl und teilt $p$ auch das Produkt $xy$ der zwei Zahlen $x$ und $y$ nicht, so teilt $p$ auch nicht einen der beiden Faktoren $x$ oder $y$.}\\

e) Hans und Franz sind verwandt.\\

\textit{Hans und Franz sind nicht verwandt.}\\

f) Hans und Franz sind rothaarig.\\

\textit{Weder Hans noch Franz sind rothaarig.}\\

g) Jede Stadt, die an einem Fluss liegt, liegt entweder am Rhein oder an der Donau.\\

\textit{Liegt eine Stadt nicht an einem Fluss, so liegt sie nicht am Rhein oder an der Donau.}\\

h) Ist $G$ eine abelsche Gruppe, so ist $G$ eine zyklische Gruppe oder ein Produkt von zyklischen Gruppen.\\

\textit{Ist $G$ abelsche Gruppe, so ist $G$ keine zyklishe Gruppe oder ein Produkt von zyklischen Gruppen.}

