\section*{Aufgabe 6}

Versuchen Sie, die folgenden Aussagen auf ihre prädikatenlogische Struktur hin zu analysieren. Zerlegen Sie die Aussagen in elementare Aussageformen, die mit aussagenlogischen Verknüpfungen und/oder prädikatenlogischen Quantoren zusammengesetzt sind.\\

Beachten Sie, dass es hierbei meist nätig ist, die Aussagen in drastischer Weise umzuformulieren, um die formale prädikatenlogische Struktur sichtbar zu machen.\\

a) Jede natürliche Zahl ist eine Primzahl oder eine gerade Zahl.\\

$\forall x ((x \text{ ist eine natürliche Zahl}) \Rightarrow (x \text{ ist eine Primzahl}) \lor (x \text{ ist eine gerade Zahl}))$\\

b) Es gibt eine reelle Zahl $x$ mit der Eigenschaft $3x^2 + 5x - 7 = 0$.\\

$\exists x (P(x))$\\

c) Die Gleichung $x^3 - 3x = 7$ ist lösbar.\\

$\exists x (P(x))$\\

d) Für je zwei reelle Zahlen $a$ und $b$ ist die Gleichung $ax + b = 0$ lösbar.\\

$\forall ab (P(a,b))$