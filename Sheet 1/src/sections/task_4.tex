\section*{Aufgabe 4}

Es seien A; B; C Aussagen.\\

Formulieren Sie die folgenden Aussagen so, dass ihr logischer Gehalt gleich bleibt, dass aber nur die Junktoren 'und', 'oder', 'nicht' und 'wenn \dots dann' Verwendung finden.

a) Zwar A, aber nicht B.\\

\textit{A und nicht B}\\

b) Nur dann, wenn A gilt, gilt auch B.\\

\textit{wenn A dann B und wenn B dann A}\\

\begin{tabular}{c|c|c}
A & B & A $\Leftrightarrow$ B\\
\hline
0 & 0 & 1\\
0 & 1 & 0\\
1 & 0 & 0\\
1 & 1 & 1\\
\end{tabular}\\~\\

c) Sowohl A als auch B.\\

\textit{A und B}\\

d) Genau dann A, wenn B.\\

\textit{wenn A dann B und wenn B dann A}\\

\begin{tabular}{c|c|c}
A & B & A $\Leftrightarrow$ B\\
\hline
0 & 0 & 1\\
0 & 1 & 0\\
1 & 0 & 0\\
1 & 1 & 1\\
\end{tabular}\\~\\

e) Wenn A, dann gilt B nur, falls C nicht gilt.\\

\textit{Wenn nicht C und A  dann B} ($(\lnot C \land A) \Rightarrow B$)\\

f) C gilt höchstens dann, wenn sowohl A als auch B gilt.\\

\textit{Wenn (A und B) dann C}\\