\section*{Aufgabe 4}

Es seien A; B; C Aussagen.\\

Formulieren Sie die folgenden Aussagen so, dass ihr logischer Gehalt gleich bleibt, dass aber nur die Junktoren 'und', 'oder', 'nicht' und 'wenn \dots dann' Verwendung finden.

a) Zwar A, aber nicht B.\\

\textit{A und nicht B}\\

b) Nur dann, wenn A gilt, gilt auch B.\\

\textit{Wenn A dann B}\\

c) Sowohl A als auch B.\\

\textit{A und B}\\

d) Genau dann A, wenn B.\\

\textit{Wenn B dann A}\\

e) Wenn A, dann gilt B nur, falls C nicht gilt.\\

\textit{Wenn A dann B und nicht (B und C)}\\

f) C gilt höchstens dann, wenn sowohl A als auch B gilt.\\

\textit{Wenn (A und B) dann C}\\