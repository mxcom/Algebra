\section*{Aufgabe 2}

Versuchen Sie, die folgenden Aussagen logisch zu zerlegen in elementarere Aussagen, die mit aussagenlogischen Operatoren (siehe oben) verknüpft sind!\\

Achten Sie darauf, dass es für diese logische Zerlegung u.U. nötig ist, die Aussagen anders zu formulieren, ohne an ihrer eigentlichen (logischen) Bedeutung etwas zu ändern.\\

Kursiv gedruckt: \textit{Elementare Aussage}\\
Fett gedruckt: \textbf{Aussagenlogische Operatoren (Junktoren)}\\

a) Köln und Düsseldorf liegen am Rhein.\\

\textit{Köln liegt am Rhein} \textbf{und} \textit{Düsseldorf liegt am Rhein}.\\

b) Eine der Städte Frankfurt, Köln, Dresden und Berlin liegt an der Elbe.\\

\textit{Frankfurt liegt an der Elbe} \textbf{oder} \textit{Köln liegt an der Elbe} \textbf{oder} \textit{Dresden liegt an der Elbe} \textbf{oder} \textit{Berlin liegt an der Elbe}\\

c) Wenn es regnet, werden die Straßen nass und der Himmel verdunkelt sich.\\

A: "\textit{Es regnet}"\\
B: "\textit{Die Straßen werden nass}"\\
C: "\textit{Der Himmel verdunkelt sich}"\\~\\

\textbf{Wenn} A \textbf{dann} (B \textbf{und} C)\\

d) Ist $p$ eine Primzahl und teilt $p$ das Produkt $xy$ der zwei Zahlen $x$ und $y$, so teilt $p$ auch einen der beiden Faktoren $x$ oder $y$.\\

A: "\textit{p ist eine Primzahl}"\\
B: "\textit{p teilt xy}"\\
C: "\textit{p teilt x}"\\
D: "\textit{p teilt y}"\\

\textbf{Wenn} (A und B) \textbf{dann} (C \textbf{oder} D) \textbf{und nicht} (C \textbf{und} D)\\

e) Hans und Franz sind verwandt.\\

Bereits elementar Aussage ('Und' ist hier kein elementar logischer Operator).\\

f) Hans und Franz sind rothaarig.\\

\textit{Hand ist rothaaring} \textbf{und} \textit{Franz ist rothaarig}.\\

g) Jede Stadt, die an einem Fluss liegt, liegt entweder am Rhein oder an der Donau.\\

A: "\textit{Die Stadt liegt an einem Fluss}"\\
B: "\textit{Die Stadt liegt am Rhein}"\\
C: "\textit{Die Stadt liegt an der Donau}"\\

\textbf{Wenn} A \textbf{dann} (B \textbf{oder} C) \textbf{und nicht} (B \textbf{und} C)\\

h) Ist $G$ eine abelsche Gruppe, so ist $G$ eine zyklische Gruppe oder ein Produkt von zyklischen Gruppen.\\

A: "\textit{G ist eine abelsche Gruppe}"\\
B: "\textit{G ist eine zyklische Gruppe}"\\
C: "\textit{G ist Produkt von zyklischer Gruppe}"\\

\textbf{Wenn} A \textbf{dann} (B \textbf{oder} C)

