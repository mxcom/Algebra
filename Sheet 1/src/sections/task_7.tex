\section*{Aufgabe 7}

Es seien $P(n); Q(n); R(n); S(n); T(n)$ die folgenden Aussageformen:\\

\begin{tabular}{lcl}
$P(n)$ & : & $n$ ist eine gerade natürliche Zahl\\
$Q(n)$ & : & $n$ ist eine Primzahl\\
$R(n)$ & : & $n$ ist Summe zweier Primzahlen\\
$S(n)$ & : & es gibt eine natürliche Zahl $k > 1$ mit $n = 2k - 1$\\
$T(n)$ & : & $n$ ist größer als $2$\\~\\
\end{tabular}

Formulieren Sie die folgenden prädikatenlogischen Formeln umgangssprachlich aus:\\

a) $\forall n ((Q(n) \land T(n)) \Rightarrow (\lnot P(n))$\\

\textit{Für alle n gilt, wenn n eine Primzahl ist und n größer als zwei ist, dann ist n keine gerade natürliche Zahl.}\\

b) $\forall n ((T(n) \land P(n)) \Rightarrow (R(n)))$\\

\textit{Für alle n gilt wenn n größer als zwei ist und n eine gerade natürliche Zahl ist, dann ist n die Summe zweier Primzahlen.}\\

c) $\exists n (S(n) \land \lnot R(n) \Rightarrow \lnot Q(n))$\\

\textit{Es existiert mindestens ein n was wenn es eine natürliche Zahl k > 1 mit n = 2k - 1 gibt und n nicht die Summe zweier Primzahlen ist, n keine Primzahl ist.}\\

d) $\lnot \forall n (S(n) \Rightarrow Q(n))$\\

\textit{Es existiert kein n was, wenn es eine natürliche Zahl k > 1 mit n = 2k - 1 gibt, eine Primzahl ist.}\\

e) $\exists n (R(n) \land Q(n))$\\

\textit{Es existiert mindestens ein n, was die Summe zweier Primzahlen ist und selber eine Primzahl ist.}