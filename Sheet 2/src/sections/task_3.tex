\section*{Aufgabe 3}

\begin{enumerate}[label={a)}, leftmargin=*]
\item Zeigen Sie, daß die folgende aussagenlogische Formel eine Tautologie ist:\\
$(A \lor B) \Rightarrow (\lnot B \Rightarrow A)$
\end{enumerate}

\begin{table}[h]
\centering
\begin{tabular}{c|c|c|c|c|c}
$A$ & $B$ & $\overbrace{A \lor B}^{C}$ & $\overbrace{B \lor A}^{D}$ & $\lnot C$ & $\lnot C \lor D$\\
\hline
1 & 1 & 1 & 1 & 0 & 1\\
1 & 0 & 1 & 1 & 0 & 1\\
0 & 1 & 1 & 1 & 0 & 1\\
0 & 0 & 0 & 0 & 1 & 1\\
\end{tabular}
\end{table}

\textit{Tautologie}

\begin{enumerate}[label={b)}, leftmargin=*]
\item Formulieren Sie die Aussage, die aus der obigen Formel entsteht, wenn die aussagenlogischen Variablen A und B ersetzt werden durch die elementaren Aussagen:\\
    ``Fipi lüllt'' bzw. ``Gluka urzt''.\\
    Überzeugen Sie sich, dass diese Aussage auch im intuitiven Sinne wahr ist.
\end{enumerate}

\textit{``Wenn Fipi lüllt oder Gluka urzt dann hat, wenn Gluka nicht geurzt hat, Fippi gelüllt.''}

\begin{enumerate}[label={c)}, leftmargin=*]
\item Ersetzen Sie nun die Variablen A und B durch die (nicht mehr aussagenlogisch elementaren)
Aussagen:\\
``Knull ixt oder Gluka pramelt'' bzw. ``Falls Akru urzt, so watzelt das Dapi''.
\end{enumerate}

\textit{``Wenn Knull ixt oder Gluka pramelt oder wenn Akru urzt dann watzelt Dapi, dann wenn Akru urzt, und Dapi nicht watzelt, dann ixt Knull oder Gluka pramelt.''}

Versuchen Sie, die Wahrheit auch dieser (sprachlich sicherlich unübersichtlichen) Aussage sich intuitiv klar zu machen.