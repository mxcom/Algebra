\section*{Aufgabe 1}

Welche der folgenden Zeichenketten sind aussagenlogische Formeln?\\

a) $((p) \land (\lnot(q))) \Rightarrow (((p) \lor ((r) \lor (q))) \land (\lnot (p)))$\\

\textit{Es ist eine aussagenlogische Formel.}\\~\\~\\~\\~\\~\\~\\~\\

b) $((a) \lor (\lnot (b)) \land (\lnot (c)) \Rightarrow (a) \land (\lnot(c))))$\\

\textit{Es ist keine aussagenlogische Formel (Letzte Klammer ist falsch/zu viel).}\\~\\~\\~\\~\\~\\~\\~\\

c) $((p) \Rightarrow (\lnot (r))) \lor (\lnot (q) \land ((r) \land (\lnot (p))))$\\

\textit{Es ist eine aussagenlogische Formel.}\\~\\~\\~\\~\\~\\~\\~\\

d) $(\lnot (p)) \lor ((q) \land (q)) \lor ((p)(r) \Rightarrow (r))$\\

\textit{Es ist keine aussagenlogische Formel ist keine valide Formel}\\~\\~\\~\\~\\~\\~\\~\\

e) $((a) \lor (b)) \Rightarrow (((c) \land (a)) \lor ((a) \Rightarrow (b)))$\\

\textit{Es ist eine aussagenlogische Formel (Alle Klammern richtig).}\\~\\~\\~\\~\\~\\~\\~\\

Stellen Sie für die korrekten aussagenlogischen Formeln jeweils einen Formelbaum auf. Für die
Zeichenketten, die keine korrekten Formeln darstellen, sollten Sie allgemeine Sätze formulieren, die für eine aussagenlogische Formel stets erfüllt sind, die aber von der gegebenen Zeichenkette nicht erfüllt wird.