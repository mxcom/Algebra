\section*{Aufgabe 5}

Die folgende Formel charakterisiert die Stetigkeit der Funktion $f$ in einem Punkt $x_0$:\\
$\forall \epsilon > 0 \ \exists \delta > 0 \ \forall  x : ((| x - x_0 | < \delta) \Rightarrow (| f(x) - f(x_0)| < \epsilon))$.

\begin{enumerate}[label={a)}, leftmargin=*]
\item ``Übersetzen'' Sie die Formel in einen natürlichen Satz.\\
(Hineweis: $\epsilon$ : Epslion, $\delta$ : Delta, | \dots | : Betrag, $f(x)$ : ``f von x'')
\end{enumerate}

\textit{``Für alle Epsilon größer 0, existiert ein delta größer 0, sodass für alle x gilt, dass falls der Betrag von $x$ minus $x_0$ größer Delta ist, Der Betrag von f von x minus f von $x_0$ kleiner Epsilon ist.''}

\begin{enumerate}[label={b)}, leftmargin=*]
\item Bilden Sie formal die Negation dieser Aussage und schreiben Sie diese so, dass
\begin{itemize}
\item kein Negationszeichen vor Quantoren steht und
\item kein Negationszeichen vor einer Implikation steht.
\end{itemize}
\end{enumerate}

\begin{enumerate}[leftmargin=*]
\item \textit{Zuerst wird die komplette Formel negiert.}
\end{enumerate}

\begin{center}
$\lnot (\forall \epsilon > 0 \ \exists \delta > 0 \ \forall  x : ((| x - x_0 | < \delta) \Rightarrow (| f(x) - f(x_0)| < \epsilon)))$
\end{center}


\begin{enumerate}[label={2.}, leftmargin=*]
\item \textit{Durch die Negation der Formel, wird jedes $\exists$ zu einem $\forall$ und jedes $\forall$ zu einem $\exists$ und die Aussage negiert sich.}
\end{enumerate}

\begin{center}
$\exists \epsilon > 0, \ \forall \delta > 0, \ \exists x : \lnot ((| x - x_0 | < \delta) \Rightarrow (| f(x) - f(x_0)| < \epsilon))$
\end{center}

\begin{enumerate}[label={3.}, leftmargin=*]
\item \textit{Da laut Aufgabenstellung vor einer Implikation keine Negation stehen soll wird die Implikation umgewandelt ($A \Rightarrow B \Leftrightarrow \lnot A \lor B$)}
\end{enumerate}

\begin{center}
$\exists \epsilon > 0, \ \forall \delta > 0, \ \exists x : \lnot (\lnot (|x - x_0| < \delta) \lor (|f(x) - f(x_0)| < \epsilon)$
\end{center}

\begin{enumerate}[label={4.}, leftmargin=*]
\item \textit{Die Negation vor der Klammer kann entfernt werden indem man es in die Klammer holt und den aussagenlogischen Operator umdreht ($\lnot(A \lor B) \Leftrightarrow \lnot A \land \lnot B$ (De Morgan))}
\end{enumerate}

\begin{center}
$\exists \epsilon > 0, \ \forall \delta > 0, \ \exists x : ((|x - x_0| < \delta) \land \lnot (|f(x) - f(x_0)| < \epsilon)$
\end{center}

\begin{enumerate}[label={5.}, leftmargin=*]
\item \textit{Die Negation vor $|f(x) - f(x_0)| < \epsilon$ kann entfernt indem man das $<$ zu einem $\geq$ macht (ist etwas nicht kleiner als was anderes, impliziert es das es größer oder gleich ist)}
\end{enumerate}

\begin{center}
$\exists \epsilon > 0, \ \forall \delta > 0, \ \exists x : ((|x - x_0| < \delta) \land (|f(x) - f(x_0)| \geq \epsilon)$
\end{center}

\begin{enumerate}[label={c)}, leftmargin=*]
\item Versuchen Sie sowohl die gegebene Aussage, als auch die negierte Aussage verbal zu formulieren.
\end{enumerate}