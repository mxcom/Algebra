\section*{Aufgabe 5}

Die folgende Formel charakterisiert die Stetigkeit der Funktion $f$ in einem Punkt $x_0$:\\
$\forall \epsilon > 0 \ \exists \delta > 0 \ \forall  x : ((| x - x_0 | < \delta) \Rightarrow (| f(x) - f(x_0)| < \epsilon))$.

\begin{enumerate}[label={a)}, leftmargin=*]
\item ``Übersetzen'' Sie die Formel in einen natürlichen Satz.\\
(Hineweis: $\epsilon$ : Epslion, $\delta$ : Delta, | \dots | : Betrag, $f(x)$ : ``f von x'')
\end{enumerate}

\textit{``Für alle Epsilon größer 0, existiert ein delta größer 0, sodass für alle x gilt, dass falls der Betrag von $x$ minus $x_0$ größer Delta ist, Der Betrag von f von x minus f von $x_0$ kleiner Epsilon ist.''}