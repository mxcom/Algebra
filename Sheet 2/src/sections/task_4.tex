\section*{Aufgabe 4}

Welche der folgenden Aussagen für eine aussagenlogische Formel A ist richtig, welche falsch (Begründung oder Gegenbeispiel)?

\begin{enumerate}[label={a)}, leftmargin=*]
\item Ist $A$ eine Tautologie, so ist $A$ erfüllbar.
\end{enumerate}

\textit{Richtig. A ist eine Tautologie, d.h. es ist immer wahr, also erfüllbar, da es für erfüllbar mindestens einmal wahr sein muss.}

\begin{enumerate}[label={b)}, leftmargin=*]
\item Ist $A$ eine Tautologie, so ist $\lnot A$ unerfüllbar.
\end{enumerate}

\textit{Richtig. Unerfüllbar heißt, dass es für alle Belegungen falsch ist, es handelt sich also um eine Kontradiktion. Wenn $A$ eine Tautologie ist, also immer wahr ist, ist das Gegenteil $\lnot A$, immer falsch, also unerfüllbar.}

\begin{enumerate}[label={c)}, leftmargin=*]
\item Ist $A$ keine Tautologie, so ist sowohl $A$ als auch $\lnot A$ erfüllbar.
\end{enumerate}

\textit{Falsch. Da für $A$, wenn es keine Tautologie ist, einen Fall geben kann, bei welchem $A$ es immer 0 ist. Bei diesem Fall ist $\lnot A$ zwar erfüllbar, $A$ ist allerdings nicht erfüllbar.}

\begin{enumerate}[label={d)}, leftmargin=*]
\item $A$ ist genau dann erfüllbar, wenn $\lnot A$ keine Tautologie ist.
\end{enumerate}

\textit{Richtig. Wenn $\lnot A$ keine Tautologie ist, dann existiert mindestens eine 0. Somit existiert für $A$ mindestens eine 1, und ist somit erfüllbar.}

\begin{enumerate}[label={e)}, leftmargin=*]
\item Ist $A$ erfüllbar und $\lnot A$ erfüllbar, so ist $A$ keine Tautologie.
\end{enumerate}

\textit{Richtig. Wenn $A$ und $\lnot A$ erfüllbar sind, muss in $A$ mindestens eine 0 und eine 1 existieren damit auch $\lnot A$ erfüllbar ist. Somit kann $A$ keine Tautologie sein.}

\begin{enumerate}[label={f)}, leftmargin=*]
\item Ist $\lnot A$ widerlegbar, so ist $A$ eine Tautologie.
\end{enumerate}

\textit{Falsch. Widerlegbar heißt das mindestens eine 0 exitiert. D.h bei $\lnot A$ können auch einsen vorkommen, wodurch es Fälle in $A$ geben kann bei denen es nicht immer wahr ist, also keine Tautologie ist.}