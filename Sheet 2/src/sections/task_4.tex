\section*{Aufgabe 4}

Welche der folgenden Aussagen für eine aussagenlogische Formel A ist richtig, welche falsch (Begründung oder Gegenbeispiel)?

\begin{enumerate}[label={a)}, leftmargin=*]
\item Ist $A$ eine Tautologie, so ist $A$ erfüllbar.
\end{enumerate}

Wenn A wahr ist dann ist A auch 

\begin{enumerate}[label={b)}, leftmargin=*]
\item Ist $A$ eine Tautologie, so ist $\lnot A$ unerfüllbar.
\end{enumerate}


\begin{enumerate}[label={c)}, leftmargin=*]
\item Ist $A$ keine Tautologie, so ist sowohl $A$ als auch $\lnot A$ erfüllbar.
\end{enumerate}


\begin{enumerate}[label={d)}, leftmargin=*]
\item $A$ ist genau dann erfüllbar, wenn $\lnot A$ keine Tautologie ist.
\end{enumerate}


\begin{enumerate}[label={e)}, leftmargin=*]
\item Ist $A$ erfüllbar und $\lnot A$ erfüllbar, so ist $A$ keine Tautologie.
\end{enumerate}


\begin{enumerate}[label={f)}, leftmargin=*]
\item Ist $\lnot A$ widerlegbar, so ist $A$ eine Tautologie.
\end{enumerate}