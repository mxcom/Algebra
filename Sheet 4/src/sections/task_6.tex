\section*{Aufgabe 6}

Es sei $M = \{ n \in \mathbb{N} \ | \ 1 \leq n \leq 13 \}$. Auf $M^2$ betrachten wir folgende Relation $S$: $(a,b)S(a',b') \Leftrightarrow a + b' = a' + b$\\

a) Man zeige, dass $S$ eine Äquivalenzrelation ist.\\

\textit{Eine Äquivalenzrelation hat die Eigenschaften Reflexivität, Symmetrie und Trasitivität.}\\

\textit{Reflexiv:}\\
\textit{$(a,b)S(a,b)$. Für $(a,b)S(a,b)$ gilt $a + b = a + b$. Da die linke Seite gleich rechte Seite ist stehen sie in Relation und es ist somit reflexiv.}\\

\textit{Symmetrie:}\\
\textit{$(a,b)S(a',b') \Rightarrow (a',b')S(a,b)$. Für $(a,b)S(a',b')$ gilt $a + b' = a' + b$ und für $(a',b')S(a,b)$ gilt $a' + b = a + b'$. Somit ist es symmetrisch.}\\

\textit{Transitiv:}\\
\textit{$(a,b)S(c,d) \land (c,d)S(e,f) \Rightarrow (a,b)S(e,f)$. Für $(a,b)S(c,d)$ gilt $a + d = c + b$ und für $(c,d)S(e,f)$ gilt $c + f = e + d$. Nicht transitiv ???}\\

b) Man gebe die Äquivalenzklassen von $(2,5)$ und von $(7,3)$ explizit an.\\

\textit{Für die Äquivalenzklasse von $x = (2,5)$ gilt $[(2,5)]_R = \{ y \in A \ | \ (2,5)Ry \}$. D.h wir suchen alle Tupel für die die Gleichung $2 + b' = a' + 5$ aufgeht.}\\

$[(2,5)]_R = \{ (1,4), (2,5), (3,6), (4,7), (5,8), (6,9), (7,10), (8,11), (9,12), (10,13) \}$\\

\textit{Für die Äquivalenzklasse von $x = (7,3)$ gilt $[(7,3)]_R = \{ y \in A \ | \ (7,3)Ry \}$. D.h wir suchen alle Tupel für die die Gleichung $7 + b' = a' + 3$ aufgeht.}\\

$[(7,3)]_R = \{ (5,1), (6,2), (7,3), (8,4), (9,5), (10,6), (11,7), (12,8), (13,9) \}$\\

\newpage