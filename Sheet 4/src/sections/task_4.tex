\section*{Aufgabe 4}

Es sei $M \subseteq \mathbb{N}$ eine beliebige Teilmenge.\\
Auf $M$ betrachten wir die ``Teilrelation'' $T$:\\

$xTy \Leftrightarrow x \ ist \ Teiler \ von \ y$\\

a) Man zeige, dass $T$ eine Ordnungsrelation ist.\\

\textit{Reflexiv:}\\
\textit{$xTx \rightarrow$ jede Zahl ist Teiler von sich selbst.}\\

\textit{Antisymmetrie:}\\
\textit{Wenn $xTy \land yTx$ gelten muss, müssen zwei Zahlen $n,m \in \mathbb{N}$ exitieren, sodass $x = n \cdot y$ und $y = m \cdot x$ ist. Somit gilt $x = n \cdot y = n \cdot (m \cdot x) = (n \cdot m) \cdot y$. Dies kann allerdings nur stimmen wenn $m,n = 1$ sind, woraus folgt $x = y$.}\\

\textit{Transitiv:}\\
\textit{Wenn $xTy \land yTz \Rightarrow xTz$ gelten muss, müssen zwei Zahlen $n,m \in \mathbb{N}$ exitieren, sodass $y = n \cdot x$ und $z = m \cdot y$ ist. Somit gilt $z = m \cdot y = m \cdot (n \cdot x) = (m \cdot n) \cdot x$. Also gibt es eine Zahl $(m \cdot n)$ die multipliziert mit mit $x$ gleich $z$ ist. Somit ist $x$ auch Teiler von $z$.}\\

b) Es sei jetzt $M = \{2 \ | \ n \in \mathbb{N}\}$ die Menge aller geraden natürlichen Zahlen. Man zeige, dass die Relation $T$ nicht linear ist.\\

\textit{Anmerkung: Lineare Ordnung = Totalordnung}\\

$M = \{2, 4, 6, 8, 10, \dots\}$