\section*{Aufgabe 1}

Sei $A = \{x \in \mathbb{Z} \ | \ -3 \leq x \leq 3\}$\\

Geben Sie die folgenden Relationen in $A \times A$ in aufzählender Schreibweise an:\\

a) $R_1 = \{(x,y) \ | \ y < 2x + 2\}$\\

$R_1 = \{(-2,-3), (-1,-3), (-1,-2), (-1,-1), (0,-3), (0,-2), (0,-1), (0,0), (0,1), (1,-3),$

\hspace{1.3cm}$(1,-2), (1,-1), (1,0), (1,1), (1,2), (1,3), (2,-3), (2,-2), (2,-1), (2,0), (2,1), (2,2),$

\hspace{1.3cm}$(2,3), (3,-3), (3,-2), (3,-1), (3,0), (3,1), (3,2), (3,3)\}$\\

$R_{1}^{-1} = \{(-3,-2), (-3,-1), (-2,-1), (-1,-1), (-3,0), (-2,0), (-1,0), (0,0), (1,0), (-3,1),$

\hspace{1.5cm}$(-2,1), (-1,1), (0,1), (1,1), (2,1), (3,1), (-3,2), (-2,2), (-1,2), (0,2), (1,2), (2,2),$

\hspace{1.5cm}$(3,2), (-3,3), (-2,3), (-1,3), (0,3), (1,3), (2,3), (3,3)\}$\\~\\

b) $R_2 = \{(x,y) \ | \ 2x + y > 1\} \cup \{(x,y) \ | \ x = y\}$\\

\textit{Da $2x + y > 1$ sein muss kann man bereits alle Tupel welche im Bereich $x = [-3, 0]$ und $y = [-3, 1]$ liegen ausschließen da diese niemals größer 1 werden.}\\

$R_2 = \{(0,2), (0,3), (1,1), (1,2), (1,3), (2,1), (2,2), (2,3), (3,1), (3,2), (3,3)\} \ \cup$

\hspace{1.05cm}$\{(-3,-3), (-2,-2), (-1,-1), (0,0), (1,1), (2,2), (3,3)\}$

\hspace{0.63cm}$ = \{(-3,-3), (-2,-2), (-1,-1), (0,0), (0,2), (0,3), (1,1), (1,2), (1,3), (2,1),$

\hspace{1.3cm}$(2,2), (2,3), (3,1), (3,2), (3,3)\}$\\

$R_{2}^{-1} = \{(-3,-3), (-2,-2), (-1,-1), (0,0), (2,0), (3,0), (1,1), (2,1), (3,1), (1,2),$

\hspace{1.5cm}$(2,2), (3,2), (1,3), (2,3), (3,3)\}$\\~\\

c) $R_3 = \{(a,b) \ | \  3 - a^2 \leq b\} \cap \{(x,y) \ | \ x^2 = y + 3 \}$\\

$R_3 = \{(-3,-3), (-3,-2), (-3,-1), (-3,0), (-3,1), (-3,2), (-3,3), (-2,-1), (-2,0),$

\hspace{1.3cm}$(-2,1), (-2,2), (-2,3), (-1,2), (-1,3), (0,3), (1,2), (1,3), (2,-1), (2,0), (2,1), (2,2),$

\hspace{1.3cm}$(2,3), (3,-3), (3,-2), (3,-1), (3,0), (3,1), (3,2), (3,3)\} \ \cap$

\hspace{1.05cm}$\{(-2,1), (-1,-2), (0,-3), (1,-2), (2,1)\}$

\hspace{0.63cm}$ = \{(-2,1), (-1,-2), (2,1)\}$\\

$R_{3}^{-1} = \{(1,-2), (-2,-1), (1,2)\}$\\

Bestimmen Sie für alle drei Relationen jeweils die Umkehrrelation ebenfalls in aufzählender Schreibweise.

\newpage