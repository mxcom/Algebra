\section*{Aufgabe 4}

Es seien $M$, $N$, $K$ beliebige Mengen. Beweisen Sie folgende Aussagen:\\

a) $(N \ \backslash \ M) \ \backslash \ K = N \ \backslash \ (M \cup K)$\\

\begin{align*}
(N \ \backslash \ M) \ \backslash \ K &= N \ \backslash \ (M \cup K)\\
x \in ((N \land \lnot M) \land \lnot K) &\Leftrightarrow x \in (N \land \lnot (M \lor K))\\
(x \in N \land \lnot (x \in M)) \land \lnot (x \in K) &\Leftrightarrow x \in N \land \lnot (x \in M \lor x \in K))\\
x \in N \land (\lnot (x \in M) \land \lnot (x \in K)) &\Leftrightarrow x \in N \land \lnot (x \in M \lor x \in K))\\
x \in N \land \lnot (x \in M \lor x \in K)) &\Leftrightarrow x \in N \land \lnot (x \in M \lor x \in K))\\
x \in (N \land \lnot (M \lor K)) &\Leftrightarrow x \in (N \land \lnot (M \lor K))\\
N \ \backslash \ (M \cup K) &= N \ \backslash \ (M \cup K)
\end{align*}
\begin{FlushRight}
$\Box$
\end{FlushRight}

b) $N \ \backslash \ (N \ \backslash \ M) = M \Leftrightarrow M \subseteq N$\\

Wenn wir zeigen können, dass $N \ \backslash \ (N \ \backslash \ M) = M$, auch als $N = M$ verstanden werden kann, dann stimmt $N = M \Leftrightarrow M \subseteq N$, da jede Teilmenge auch immer sich selbst als Teilmenge hat.

\newpage