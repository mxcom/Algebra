\section*{Aufgabe 10}

Sind $M$ und $N$ beliebige Mengen, so definiert man die ``symmetrische Differenz'' $M \bigtriangleup N$ als die Menge aller Elemente, die in genau einer der beiden Mengen enthalten sind.\\

a) Drücken Sie den Operator $\bigtriangleup$ durch die Operatoren $\cup$, $\cap$, $\backslash$ aus.\\

$M \bigtriangleup N = (M \cup N) \ \backslash \ (M \cap N)$\\

b) Welcher logischen Operation entspricht dieser Operator?\\

\textsc{Kontravalenz/Exklusives Oder}\\

c) Bestimmen Sie die folgenden Mengen (in aufzählender Schreibweise):

\begin{itemize}
\item $\{0, 2, 4\} \bigtriangleup \{1, 2, 3, 4\}$
\end{itemize}

$\{0, 2, 4\} \bigtriangleup \{1, 2, 3, 4\} = \{ 0, 1, 3 \}$

\begin{itemize}
\item $\{1, 3, 5\} \bigtriangleup \{2, 4\}$
\end{itemize}

$\{1, 3, 5\} \bigtriangleup \{2, 4\} = \{ 1, 2, 3, 4, 5 \}$

\begin{itemize}
\item $\{ (1,2), (1,3) (1,4), (2,4) \} \bigtriangleup \{ (2,1), (2,4), (1,3) \}$
\end{itemize}

$\{ (1,2), (1,3) (1,4), (2,4) \} \bigtriangleup \{ (2,1), (2,4), (1,3) \} = \{ (1,2) (1,4), (2,1) \} $

\begin{itemize}
\item $\mathcal{P}( \{0,2\} ) \bigtriangleup \mathcal{P}( \{1, 2, 3\} )$
\end{itemize}

$\mathcal{P}( \{0,2\} ) \bigtriangleup \mathcal{P}( \{1, 2, 3\} ) = \{ \{ \ \}, \{0\}, \{2\}, \{0, 2\}\} \bigtriangleup \{ \{ \ \}, \{1\}, \{2\}, \{3\}, \{1, 2\} \{1, 3\}, \{2, 3\}, \{1, 2, 3\} \}$\\
\hspace*{10.26em}$= \{ \{0\}, \{1\}, \{3\}, \{0,2\}, \{1,2\}, \{1,3\}, \{2,3\}, \{1,2,3\} \}$

\newpage
