\section*{Aufgabe 9}

Untersuchen Sie, ob die folgenden Gleichungen für beliebige Mengen $M, N$ gelten:\\

a) $\mathcal{P}(N) \cap \mathcal{P}(M) = \mathcal{P}(N \cap M)$\\

Die Schnittmenge ist definiert als: $A \cap B := \{x \ | \ x \in A \land x \in B \}$\\

Zu zeigen ist, (I) $x \in \mathcal{P}(N) \cap \mathcal{P}(M) \subseteq \mathcal{P}(N \cap M)$ und (II) $\mathcal{P}(N) \cap \mathcal{P}(M) \supseteq x \in \mathcal{P}(N \cap M)$

\begin{enumerate}[leftmargin=2em]
\item[(I)] $x \in \mathcal{P}(N) \cap \mathcal{P}(M)$\\
Laut Schnittmengendefinition gilt $x \in P(N) \land x \in P(M)$.\\
Somit ist $x \subseteq N \land x \subseteq M$, also $x \subseteq N \cap M$ und damit $x \in P(N \cap M)$.
\item[(II)] $x \in \mathcal{P}(N \cap M)$\\
Laut Potenzmengendefinition gilt $x \subseteq N \cap M$.\\
Somit ist $x \subseteq N \land x \subseteq M$, also $x \in P(N) \land x \in P(M)$ und damit $x \in P(N) \cap P(M)$.
\end{enumerate}\
\\

b) $\mathcal{P}(N) \cup \mathcal{P}(M) = \mathcal{P}(N \cup M)$\\

Die Vereinigungsmenge ist definiert als: $A \cap B := \{x \ | \ x \in A \lor x \in B \}$\\

Zu zeigen ist, (I) $\mathcal{P}(N) \cup \mathcal{P}(M) \subseteq \mathcal{P}(N \cup M)$ und (II) $\mathcal{P}(N) \cup \mathcal{P}(M) \supseteq \mathcal{P}(N \cup M)$

\begin{enumerate}[leftmargin=2em]
\item[(I)] $x \in \mathcal{P}(N) \cup \mathcal{P}(M)$\\
Laut Vereinigungsmengendefinition gilt $x \in P(N) \lor x \in P(M)$.\\
Somit ist $x \subseteq N \lor x \subseteq M$, also $x \subseteq N \cup M$ und damit $x \in P(N \cup M)$.
\item[(II)] $x \in´ \mathcal{P}(N \cup M)$\\
Laut Potenzmengendefinition gilt $x \subseteq N \cap M$.\\
Somit ist $x \subseteq N \lor x \subseteq M$, also $x \in P(N) \lor x \in P(M)$ und damit $x \in P(N) \cup P(M)$.
\end{enumerate}

\newpage