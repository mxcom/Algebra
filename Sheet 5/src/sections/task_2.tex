\section*{Aufgabe 2}

Gegeben sei die Menge $\mathbb{Z}_{10}$ zusammen mit der bekannten Multiplikation $\otimes$.\\

a) Geben Sie eine Verknüpfungstafel an.

\begin{table}[h]
\centering
\begin{tabular}{c|cccccccccc}
$\otimes$ & 0 & 1 & 2 & 3 & 4 & 5 & 6 & 7 & 8 & 9\\
\hline
0 & 0 & 0 & 0 & 0 & 0 & 0 & 0 & 0 & 0 & 0\\
1 & 0 & 1 & 2 & 3 & 4 & 5 & 6 & 7 & 8 & 9\\
2 & 0 & 2 & 4 & 6 & 8 & 0 & 2 & 4 & 6 & 8\\
3 & 0 & 3 & 6 & 9 & 2 & 5 & 8 & 1 & 4 & 7\\
4 & 0 & 4 & 8 & 2 & 6 & 0 & 4 & 8 & 2 & 6\\
5 & 0 & 5 & 0 & 5 & 0 & 5 & 0 & 5 & 0 & 5\\
6 & 0 & 6 & 2 & 8 & 4 & 0 & 6 & 2 & 8 & 4\\
7 & 0 & 7 & 4 & 1 & 8 & 5 & 2 & 9 & 6 & 3\\
8 & 0 & 8 & 6 & 4 & 2 & 0 & 8 & 6 & 4 & 2\\
9 & 0 & 9 & 8 & 7 & 6 & 5 & 4 & 3 & 2 & 1\\
\end{tabular}
\end{table}\

b) Bestimmen Sie alle bezüglich dieser Verknüfung invertierbaren Elemente $\mathbb{U}(\mathbb{Z}_{10}) \subseteq \mathbb{Z}_{10}$\\

\textit{Das inverse Element verknüpft mit dem eigentlichen Element muss gleich dem neutralen Element sein.}\\

$\mathbb{U}(\mathbb{Z}_{10}) = \{ 1, 3, 7, 9 \}$\\~\\

c) Ist $\mathbb{U}(\mathbb{Z}_{10}) \subseteq \mathbb{Z}_{10}$ bezüglich $\otimes$ abgeschlossen?

\textit{Es ist abgeschlossen, da jedes Element genau einmal in jeder Zeile und Spalte vorkommt (Sudoku).}\\~\\

d) Geben Sie eine Verknüpfungstafel für ($\mathbb{U}(\mathbb{Z}_{10})$, $\otimes$) an.\

\hspace{0.55cm}Definiert diese Verknüpfung eine Gruppenstruktur?\

\hspace{0.55cm}Zu welcher bekannten Gruppe ist (eventuell) diese Gruppe isomorph?\

\hspace{0.55cm}Geben Sie ggf. einen Isomorphismus an.\

\begin{table}[h]
\centering
\begin{tabular}{c|cccc}
$\otimes$ & 1 & 3 & 7 & 9\\
\hline
1 & 1 & 3 & 7 & 9\\
3 & 3 & 9 & 1 & 7\\
7 & 7 & 1 & 9 & 3\\
9 & 9 & 7 & 3 & 1
\end{tabular}
\end{table}

\textit{Es ist eine Gruppe:}
\begin{enumerate}
\item \textit{Ist Abgeschlossen}
\item \textit{Das neutrale Element existiert}
\item \textit{Für jedes Element existiert ein neutrales Element.}
\item \textit{Es ist assoziativ}
\end{enumerate}\

e) Lösen Sie in $\mathbb{Z}_{10}$ die Gleichung $3 \otimes x \oplus 4 = 5$\\

$\underbrace{\underbrace{3 \otimes x}_{1} \oplus 4}_{5} = 5$, damit die Gleichung aufgeht muss $x = 7$.

\newpage


