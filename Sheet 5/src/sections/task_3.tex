\section*{Aufgabe 3}

Wir betrachten die Symmetriegruppe ($\mathbb{S}_7$ , $\circ$) der Permutationen der 7 Elemente $\{1, 2, 3, 4,\\ 5, 6, 7\}$ mit der Abbildungskomposition als Verknüpfung.\\

a) Berechnen Sie für die Elemente
$\phi =
\begin{pmatrix}
1 & 2 & 3 & 4 & 5 & 6 & 7\\
2 & 3 & 5 & 6 & 1 & 4 & 7\\
\end{pmatrix}$ und
$\psi = 
\begin{pmatrix}
1 & 2 & 3 & 4 & 5 & 6 & 7\\
4 & 6 & 2 & 3 & 7 & 1 & 5\\
\end{pmatrix}$\\~\\

% phi^-1
$\phi^{-1} =
\begin{pmatrix}
1 & 2 & 3 & 4 & 5 & 6 & 7\\
5 & 1 & 2 & 6 & 3 & 4 & 7\\
\end{pmatrix}$\\~\\~\\

% phi^4
$\phi^{4} = \phi^{2} \ \circ \ \phi^{2} =
\begin{pmatrix}
1 & 2 & 3 & 4 & 5 & 6 & 7\\
3 & 5 & 1 & 4 & 2 & 6 & 7\\
\end{pmatrix} \ \circ \
\begin{pmatrix}
1 & 2 & 3 & 4 & 5 & 6 & 7\\
3 & 5 & 1 & 4 & 2 & 6 & 7\\
\end{pmatrix} =
\begin{pmatrix}
1 & 2 & 3 & 4 & 5 & 6 & 7\\
1 & 2 & 3 & 4 & 5 & 6 & 7\\
\end{pmatrix}$\\~\\~\\

% psi^3
$\psi^{3} = \psi^{2} \ \circ \ \psi^{1} = 
\begin{pmatrix}
1 & 2 & 3 & 4 & 5 & 6 & 7\\
3 & 1 & 6 & 2 & 5 & 4 & 7\\
\end{pmatrix} \ \circ \
\begin{pmatrix}
1 & 2 & 3 & 4 & 5 & 6 & 7\\
4 & 6 & 2 & 3 & 7 & 1 & 5\\
\end{pmatrix} =
\begin{pmatrix}
1 & 2 & 3 & 4 & 5 & 6 & 7\\
2 & 4 & 1 & 6 & 7 & 3 & 5\\
\end{pmatrix}$\\~\\~\\

% phi o psi^3
$\phi \ \circ \ \psi^{3} =
\begin{pmatrix}
1 & 2 & 3 & 4 & 5 & 6 & 7\\
2 & 3 & 5 & 6 & 1 & 4 & 7\\
\end{pmatrix} \ \circ \
\begin{pmatrix}
1 & 2 & 3 & 4 & 5 & 6 & 7\\
2 & 4 & 1 & 6 & 7 & 3 & 5\\
\end{pmatrix}
= 
\begin{pmatrix}
1 & 2 & 3 & 4 & 5 & 6 & 7\\
3 & 6 & 2 & 4 & 7 & 5 & 1\\
\end{pmatrix}$\\

\begin{align*}
\psi^{2} \ \circ \ \phi \ \circ \ \psi^{-1} &=
\begin{pmatrix}
1 & 2 & 3 & 4 & 5 & 6 & 7\\
3 & 5 & 1 & 4 & 2 & 6 & 7\\
\end{pmatrix} \ \circ \
\begin{pmatrix}
1 & 2 & 3 & 4 & 5 & 6 & 7\\
2 & 3 & 5 & 6 & 1 & 4 & 7\\
\end{pmatrix} \ \circ \
\begin{pmatrix}
1 & 2 & 3 & 4 & 5 & 6 & 7\\
6 & 3 & 4 & 1 & 7 & 2 & 5\\
\end{pmatrix}\\~\\
&=
\begin{pmatrix}
1 & 2 & 3 & 4 & 5 & 6 & 7\\
3 & 5 & 1 & 4 & 2 & 6 & 7\\
\end{pmatrix} \ \circ \
\begin{pmatrix}
1 & 2 & 3 & 4 & 5 & 6 & 7\\
4 & 5 & 6 & 2 & 7 & 3 & 1\\
\end{pmatrix}\\~\\
&=
\begin{pmatrix}
1 & 2 & 3 & 4 & 5 & 6 & 7\\
2 & 5 & 4 & 1 & 7 & 6 & 3\\
\end{pmatrix}
\end{align*}

\newpage

b) Bestimmen Sie die kleinste Zahl $k > 0$, sodass $\phi^{k}$ das neutrale Element der Gruppe ergibt.\\

\textit{4 und 6 tauschen somit muss $k$ eine gerade Zahl sein. 1, 2, 3, 5 werden so aufeinander abgebildet, dass es 4 Verknüpfungen braucht damit jedes Element auf sich selbst abgebildet wird, somit ist $k = 4$}\\

$1 \ \ \rightarrow \ \ 2 \ \ \rightarrow \ \ 3 \ \ \rightarrow \ \ 5 \ \ \rightarrow \ \ 1$\\
$2 \ \ \rightarrow \ \ 3 \ \ \rightarrow \ \ 5 \ \ \rightarrow \ \ 1 \ \ \rightarrow \ \ 2$\\
$3 \ \ \rightarrow \ \ 5 \ \ \rightarrow \ \ 1 \ \ \rightarrow \ \ 2 \ \ \rightarrow \ \ 3$\\
$4 \ \ \rightarrow \ \ 6 \ \ \rightarrow \ \ 4 \ \ \rightarrow \ \ 6 \ \ \rightarrow \ \ 4$\\
$5 \ \ \rightarrow \ \ 1 \ \ \rightarrow \ \ 2 \ \ \rightarrow \ \ 3 \ \ \rightarrow \ \ 5$\\
$6 \ \ \rightarrow \ \ 4 \ \ \rightarrow \ \ 6 \ \ \rightarrow \ \ 4 \ \ \rightarrow \ \ 6$\\
$\underbrace{7 \ \ \rightarrow \ \ 7}_{k = 1 \Leftrightarrow \phi} \ \ \rightarrow \underbrace{7}_{k = 2} \rightarrow \underbrace{7}_{k = 3} \rightarrow \underbrace{7}_{k = 4}$\\~\\

c) Lösen Sie in $(\mathbb{S}_7, \circ)$ die folgende Gleichung für die Unbekannte $\xi:$
$$
\phi^2 \ \circ \ \psi \ \circ \ \xi \ \circ \ \phi^3 = \psi \ \circ \ \phi^2
$$\\

d) Zeigen Sie, dass die Menge $\mathbb{T} = \{\rho \in \mathbb{S}_7 \ | \ \rho(\{1, 2, 5\}) \subseteq \{1, 2, 5\}\} \subseteq \mathbb{S}_7$ eine Untergruppe ist.\\

e) Geben Sie die von $\sigma = \begin{pmatrix}1 & 2 & 3 & 4 & 5 & 6 & 7\\2 & 4 & 5 & 3 & 1 & 7 & 6\\\end{pmatrix}$ erzeuge Untergruppe von $\mathbb{S}_7$ an. Welche Ordnung hat diese?

\newpage



