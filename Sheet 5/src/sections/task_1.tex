\section*{Aufgabe 1}

a) Erstellen Sie für die Gruppe ($\mathbb{Z}$, $\oplus$) eine Gruppentafel.\

\hspace{0.5cm}(Hinweis: $\mathbb{Z}_6 = \{ 0,1,2,3,4,5 \}$) und $\oplus$ (manchmal $\oplus _6$) bedeutet + mit \textit{mod} 6)\\

\begin{table}[h]
\centering
\begin{tabular}{c|c|c|c|c|c|c}
$\oplus$ & 0 & 1 & 2 & 3 & 4 & 5\\
\hline
0 & 0 & 1 & 2 & 3 & 4 & 5\\
\hline
1 & 1 & 2 & 3 & 4 & 5 & 0\\
\hline
2 & 2 & 3 & 4 & 5 & 0 & 1\\
\hline
3 & 3 & 4 & 5 & 0 & 1 & 2\\
\hline
4 & 4 & 5 & 0 & 1 & 2 & 3\\
\hline
5 & 5 & 0 & 1 & 2 & 3 & 4\\
\end{tabular}
\end{table}\

b) Bestimmen Sie für die Elemente 3 und 4 in $\mathbb{Z}_6$ die bezüglich $\oplus$ inversen Elemente.\\

\textit{Das inverse Element verknüpft mit dem eigentlichen Element muss gleich dem neutralen Element sein. Also $(3 + x) \ mod \ 6 = 0$, bzw. $(4 + x) \ mod \ 6 = 0$.}

\begin{itemize}[leftmargin=*]
\item \textit{Inverses von $3$: $3$}
\item \textit{Inverses von $4$: $2$}
\end{itemize}\

c) Lösen Sie in $\mathbb{Z}_6$ die Gleichung $x \oplus 4 = 1$\\

$x = 3$

\newpage