\section*{Aufgabe 6}

Es sei $K$ die Menge der Kongruenzabbildungen eines gleichseitigen Dreiecks. $K$ besteht aus den Drehungen um den Mittelpunkt (Schnittpunkt der Seitenhalbierenden) mit Winkel $0^{\circ}$, $60^{\circ}$, $120^{\circ}$, und den Spiegelungen $s1$ , $s2$, $s3$ an den Mittelsenkrechten. $K$ bildet zusammen mit der Abbildungskomposition $\circ$ eine Gruppe.\\

a) Bei jeder der Abbildungen werden die Eckpunkte $A$, $B$, $C$ des Dreiecks permutiert, durch jede der Abbildungen ist also eine Permutation der Menge $\{A, B, C\}$ bestimmt, und jede Permutation der drei Eckpunkte entsteht auch durch eine dieser Kongruenzabbildungen. Man gebe in einer Tabelle zu jeder der Kongruenzabbildungen die entsprechende Permutation an.\\

$d_1 = \begin{pmatrix}A & B & C\\A & B & C\end{pmatrix}$, $d_2 = \begin{pmatrix}A & B & C\\C & A & B\end{pmatrix}$, $d_3 = \begin{pmatrix}A & B & C\\B & C & A\end{pmatrix}$,\\~\\

$s_1 = \begin{pmatrix}A & B & C\\A & C & B\end{pmatrix}$, $s_2 = \begin{pmatrix}A & B & C\\C & B & A\end{pmatrix}$, $s_3 = \begin{pmatrix}A & B & C\\B & A & C\end{pmatrix}$\\~\\

b) Bestimmen Sie die Gruppentafel zu dieser Gruppe $(K, \circ)$.

\begin{table}[h]
\centering
\begin{tabular}{c|cccccc}
$\circ$ & $d_1$ & $d_2$ & $d_3$ & $s_1$ & $s_2$ & $s_3$\\
\hline
$d_1$ & $d_1$ & $d_2$ & $d_3$ & $s_1$ & $s_2$ & $s_3$\\
$d_2$ & $d_2$ & $d_3$ & $d_1$ & $s_3$ & $s_1$ & $s_2$\\
$d_3$ & $d_3$ & $d_1$ & $d_2$ & $s_2$ & $s_3$ & $s_1$\\
$s_1$ & $s_1$ & $s_2$ & $s_3$ & $d_1$ & $d_2$ & $d_3$\\
$s_2$ & $s_2$ & $s_3$ & $s_1$ & $d_3$ & $d_1$ & $d_2$\\
$s_3$ & $s_3$ & $s_1$ & $s_2$ & $d_2$ & $d_3$ & $d_1$\\
\end{tabular}
\end{table}\

c) Man überzeuge sich an einigen Beispielen, dass die Abbildungskomposition $\circ$ auf $K$ genau mit der Verknüpfung in der Permutationsgruppe $\mathbb{S}_{\{A,B,C\}}$ korrespondiert. Die Zordnung aus a) definiert also einen Gruppenisomorphismus von $(K, \circ)$ auf die Permutationsgruppe $(\mathbb{S}_{\{A,B,C\}}, \circ)$.\\

\newpage





